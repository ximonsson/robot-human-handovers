\section{Background}

There are many different scenarios where we could imagine we would draw advantage of a robot performing handovers to a human. In an industrial environment, to save human workers the time of fetching tools by themselves, a robot could be employed for the task of handing over the tools when needed. In another scenario a robot could be employed into helping a human with a physical disability into reaching different things around the home. Different objects require different grips and/or movements to perform the actual handover. We can not for example hand over a sharp knife the same way that we hand over a pen, nor a hammer the same way we would for a cup of hot coffee. A challenge with robot-human handovers is to get the robot to perform the handover correctly depending on the object in question. In an industrial environment the number of tools that the robot is to handover is finite, and it is possible to train it for each one the tools. Howerver in the home of someone with a disability there is almost an infinite amount of different objects that can require handovers and it would be very inefficient to train the robot for each one of them. Existing methods for determining proper handover movements relie on robot-computed configurations and preprogrammed object-specific information. These methods all require data over correct hand grip and movement connected to objects as mentioned in this article [section IV.C]. With this work we would like to propose a way for a robot to determine grasp and handover methods purely from observations, in the goal of making it able to adapt to new objects without any prior knowlegde about grasp configurations.

\section{Research Question}

In this work we want to see if we can develop a system for a robot to learn autonomisly how to handover over a series of different objects without labeling the data, and if it can apply this to novel objects that it has not seen. We will be answering questions such as if there is a way to classify handovers and if there is a link between object properties and a type of handover. Later we will see if this is applicable to new objects.

With this project we will try to answer the following questions:
\begin{itemize}
	\item Are we able to classify objects by their handover settings?
	\item Can we relate the visual aspects of an object to its class?
	\item Is this model applicable to new objects?
\end{itemize}
