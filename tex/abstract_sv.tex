Förslag på ett system för robotar att lära sig på ett autonomt semi-supervised vis egenskaper vid överlämning för olika objekt genom att observera människor, som kan senare används även till nya objekt.

Med hjälp av inspelat material på överlämningar, identifierar vi egenskaper som gör det möjligt att klassifiera objekten genom unsupervised learning. Resultaten från denna klassifiering kombineras med bilder på objekten som används till att träna ett nätverk på ett supervised vis, som lär sig att förustpå korrekt klass för ett objekt via bilddata.

Resultaten från detta arbete visar att objekt som överlämnas på liknande vis även har liknande visuella egenskaper, och med en begränsad mängd med data kan vi träna en modell som med hög träffsäkerhet ger oss inställningarna för överlämningen utav ett objekt vare sig det har påträffats tidigare eller inte.
